\documentclass[preprint,9pt]{sigplanconf}

% The following \documentclass options may be useful:
%
% 10pt          To set in 10-point type instead of 9-point.
% 11pt          To set in 11-point type instead of 9-point.
% authoryear    To obtain author/year citation style instead of numeric.

\usepackage{amsmath}
\usepackage{epsfig}
\usepackage{algorithmic}
%\usepackage{program}
\usepackage{algorithm}
%\numberwithin{algorithm}{chapter}
%\NumberProgramstrue

\begin{document}

%\conferenceinfo{XXXX '13}{Feb 30-31, San Jose.} 
%\copyrightyear{2013} 
%\copyrightdata{[to be supplied]} 

%\titlebanner{DRAFT. DO NOT REDISTRIBUTE}        % These are ignored unless
%\preprintfooter{DRAFT. DO NOT REDISTRIBUTE.}   % 'preprint' option specified.

\title{Building Resilient Cloud Over Unreliable Commodity Infrastructure}
%\subtitle{Subtitle Text, if any}

\authorinfo{xxx}{xxx}{xxx}
%\markboth{DRAFT: DO NOT REDISTRIBUTE}{DRAFT: DO NOT REDISTRIBUTE}
\maketitle

\begin{abstract}
Cloud Computing has emerged as a successful computing paradigm for efficiently utilizing managed compute infrastructure such as high speed rack-mounted servers, connected with high speed networking, and reliable storage. Usually such infrastructure is dedicated, physically secured and has reliable power and networking infrastructure. However, much of our idle compute capacity is present in unmanaged infrastructure like idle desktops, lab machines, physically distant server machines, and laptops. We present a scheme to utilize this idle compute capacity on a best-effort basis and provide high availability even in face of failure of individual components or facilities.

We run virtual machines on the commodity infrastructure and present a cloud interface to our end users. The primary challenge is to maintain availability in the presence of node failures, network failures, and power failures. We run multiple copies of a Virtual Machine (VM) redundantly on geographically dispersed physical machines to achieve availability. If one of the running copies of a VM fails, we seamlessly switchover to another running copy. We use Virtual Machine Record/Replay capability to implement this redundancy and switchover. In current progress, we have implemented VM Record/Replay for uniprocessor machines over Linux/KVM and are currently working on VM Record/Replay on shared-memory multiprocessor machines.
\end{abstract}

%\terms

%\keywords
\section{Introduction}
\label{sec:intro}
\cite{agesen_vmm_benchmarking}
\section{Virtual Machine Record Replay}
\label{sec:record_replay}
\section{Redundancy and Failover in Cloud}
\label{sec:failover}
\section{Implementation}
\label{sec:implementation}
\section{Experimental Results}
\label{sec:results}
\section{Related Work}
\label{sec:relwork}
\section{Conclusion}
\label{sec:conclusion}
\bibliographystyle{abbrvnat-custom}
\bibliography{cloud_failover_ccem12}
\end{document}
